% % % % % % % % % % % % % % % % % % % % % % % % % % % % % % % % % % % % 
% 
% FS-Vorlage											Stand: 30.01.12
%
% Formelsammlungsvorlage von Emanuel Regnath und Martin Zellner	
% Bietet verschiedene Abkürzungen und Befehle	
%
% % % % % % % % % % % % % % % % % % % % % % % % % % % % % % % % % % % % 


% Dokumenteinstellungen
% ======================================================================

% Dokumentklasse (Schriftgröße 6, DIN A4, Artikel)
\documentclass[6pt,a4paper]{scrartcl}
%\documentclass[5pt,a4paper]{scrartcl} %USE IN CASE OF EMERGENCY  geschafft! emergency not needed

% Pakete laden
\usepackage[utf8]{inputenc}		% Zeichenkodierung: UTF-8 (für Umlaute)   
\usepackage[german]{babel}		% Deutsche Sprache
\usepackage{multicol}			% ermöglicht Seitenspalten  
\usepackage{booktabs}			% bessere Tabellenlinien
\usepackage{enumitem}			% bessere Listen
\usepackage{graphicx}			% Zum Bilder einfügen benötigt
\usepackage{xcolor}
\usepackage{pbox}				%Intelligent parbox: \pbox{maximum width}{blabalbalb \\ blabal}
\usepackage{scientific}			% Eigenes Paket
\usepackage{scrtime}
\usepackage{parskip} 			%Verhindert das einrücken am Zeilenanfang
\usepackage{titlesec}

\newcommand{\mrule}{\relax}


% .:: Seitenlayout und Ränder
% ======================================================================
\usepackage{geometry}
\geometry{a4paper,landscape, left=6mm,right=6mm, top=0mm, bottom=3mm,includeheadfoot} 


% .:: Kopf- und Fußzeile
% ======================================================================
\usepackage{fancyhdr}
\pagestyle{fancy}
\fancyhf{}

   \fancyfoot[C]{von Emanuel Regnath und Martin Zellner - Mail: \emph{info@latex4ei.de}}
   \renewcommand{\headrulewidth}{0.0pt} %obere Linie ausblenden
   \renewcommand{\footrulewidth}{0.1pt} %obere Linie ausblenden

   \fancyfoot[R]{Stand: \today \ um \thistime \ Uhr \qquad \thepage}
   \fancyfoot[L]{Homepage: www.latex4ei.de -- Fehler bitte \emph{sofort} melden.}
	
% Schriftart SANS für bessere Lesbarkeit bei kleiner Schrift
\renewcommand{\familydefault}{\sfdefault} 
% Array- und Tabellenabstände vergrößern
\renewcommand{\arraystretch}{1.2}


% .:: Überschriften anpassen
% ======================================================================

%\titleformat{ command }[ shape ]{ format }{ label }{ sep }{ before-code }[ after-code ]
\titleformat{\section}{\Large \bfseries}{\thesection .}{0.5em}{}[\hrule \hrule ]
\titleformat{\subsection}{\large \bfseries}{\thesubsection .}{0.3em}{}[ ]

%\titlespacing{Überschriftart}{keine Ahnung}{Abstand oberhalb}{Abstand unterhalb}
\titlespacing{\section}{0em}{1.0em}{0.1em}
\titlespacing{\subsection}{0em}{0.2em}{-0.4em}
\titlespacing{\subsubsection}{0em}{0em}{-0.5em}

%\let\vec\oldvec

\DeclareMathOperator{\trans}{trans}
\DeclareMathOperator{\rotx}{rotX}
\DeclareMathOperator{\roty}{rotY}
\DeclareMathOperator{\rotz}{rotZ}



% Dokumentbeginn
% ======================================================================
\begin{document}


% Aufteilung in Spalten
\vspace{-4mm}
\begin{multicols}{4}
	\vspace{-20mm}{
	\parbox{2.3cm}{
		%\includegraphics[height=1.4cm]{./img/Logo.pdf}		
	}
	\parbox{4cm}{
		\emph{\huge{Einf\"uhrung in die\\Roboterregelung}}
	}}
\vspace{5mm} % Man muss optimieren wos nur geht ;)
% ---------------------------------------------------
% | 		Einführung in die Roboterregelung		|
% | ehemals: Grundlagen intelligenter Roboter		|
% ~~~~~~~~~~~~~~~~~~~~~~~~~~~~~~~~~~~~~~~~~~~~~~~~~~~
%=======================================================================

Dieses Fach wurde bis 2013 unter dem Namen ``Grundlagen intelligenter Roboter'' gelehrt.\par
\textcolor{red}{	\textbf{HINWEIS:} Die Formelsammlung ist eine einfache Mitschrift, sehr ungeordnet und kann grobe Fehler enthalten. Sie dient lediglich als Überblick zum Fach.
	Wenn jemand die FS ergänzen/überarbeiten möchte, einfach melden}



\section{Mathematik}
$\dot {\ma R} = \ma S(t) \cdot R(t)$ \quad mit $\ma S(t) = \ma S(\vec \omega(t)) = \mat{ 0 & -\omega_z & \omega_y \\ \omega z & 0 & -\omega_x \\ -\omega_y \omega x & 0}$\\
Ableitung von ${}_0 \dot{\ma R} = \ma {}_0 R_{m-1} \cdot {}_{m-1} \ma R_m$\\ 

\section{Allgemeines}
\begin{tabular}{ll}
Räumliche Anordnung & $RAN = x_1,x_2,x_3,\varphi_1,\varphi_2,\varphi_3$\\
Homogene Transformation & ${}_b \ma T_a = \mat{\ma R & \vec r \\ \vec 0^\top & s} \in \R^{4 \times 4}$\\
Weltkoordinaten & $\vec w = \mat{ \vec r \\ \vec \Omega}$\\
Gelenkkordinaten & $\vec q$\\
\end{tabular}



Frame-Konzept: jedes Objekt hat sein eigenes KOSY(Frame).\\
Richtungscosinusmatrix ${}_{A} \ma R_{B}$  bzw. ${}_{A} \ma C_{B} = \mat{ c_{xx'} & c_{xy'} & c_{xz'} \\ c_{yx'} & c_{yy'} & c_{yz'} \\ c_{zx'} & c_{zy'} & c_{zz'} }$\\
Redundant, da Basisvektoren orthogonal.\\
Self-Motion: Nur zusätzliche Freiheitsgrade werden bewegt, Effektor behält seine aktuelle RAN\\
holonom: System, welches sich lokal in alle Richtungen gleichzeitig bewegen kann % dessen $n$ Freiheitsgrade nur durch $m < n$ Zwangsbedingung beschränkt ist.\\
\\
\subsection{Transformation}
Elementare Rotationsmatrizen (sind orthogonal $\ma R^\top = \ma R^{-1}$):\\
$\mat{ 1 & 0 & 0 \\ 0 & c\theta_x & - s\theta_x \\ 0 & s\theta_x & c\theta_x }$ \ $\mat{ c\theta_y & 0 & s\theta_y \\ 0  & 1 & 0 \\ -s\theta_y & 0 & c\theta_y }$\ 
$\mat{ c\theta_z & -s\theta_z & 0 \\ s\theta_z & c\theta_z & 0 \\ 0 & 0 & 1 }$\\
Verkettung von Rotationen nicht kommutativ! $\ma R_{ges} = \ma R_n \cdots \ma R_2 \cdot \ma R_1$\\


\textbf{RPY-Winkel:} Roll($\alpha$), Pitch($\beta$), Yaw($\gamma$) um $x,y,z$-Weltachsen\\
$\ma R = \ma R_z(\gamma) \cdot \ma R_y(\beta) \cdot \ma R_x(\alpha)$\\
\textbf{Eulersche Winkel} $\Psi,\Theta,\Phi$: Drehung um die aktuellen $z,x',z''$-Achsen.\\
$\ma R = \ma R_z(\Psi) \cdot \ma R_x(\theta) \cdot \ma R_z(\phi)$\\
\\
Homogene Transformation $\tma{B}{T}{A} = \mat{\tma{B}{R}{A} & \vec r' \\ \vec f^\top & w} \in \R^{4 \times 4}$\\
mit Rotation $\ma R$, Translation $\vec r'$, Verzerrung $\vec f$ und Skalierung $w$\\
Inverse: $\tma{B}{T}{A}^{-1} = \tma{A}{T}{B} = \mat{\tma{B}{R}{A}^\top & -\tma{B}{R}{A}^\top {}_B \vec r' \\ \vec 0^\top & 1}$\\
\\
hom. Translationsmatrix: $\tma{B}{T}{A} = \mat{ \ma E & \vec o' \\ \vec 0^\top & 1}$\\
Matrixmultiplikation:\\
Von links nach rechts: Um Achse im aktuellen KOSY (Euler)\\
Von rechts nach links: Um Achse im ursprünglichen KOSY (RPY)\\


\subsection{Weltmodellierung}
Ein Frame kann Bezugsystem für beliebig viele andere Frames sein aber darf selber nur an ein einziges Bezugsystem gebunden sein.

Effektortransforamtion $\tma{0}{T}{E}$\\

\subsection{Denavit-Hartenberg-Konvetion}
Regeln, zum Aufstellen der einzelnen Frames:
\begin{enumerate}
	\item $z_n$-Achse ist Drehachse des Gelenks $n+1$ umd Winkel $\theta_{n+1}$
	\item $x_n$ ist die gemeinsame Normale von $z_{n-1}$ zu $z_n$
	\item Ergänze $y_n$ so, dass ein Rechtssystem entsteht. 
\end{enumerate}
Damit ergeben sich pro Gelenk nur 4 Parameter zur Bestimmung der RAN des $n$-ten KOSY im $(n-1)$-ten KOSY\\
$rot(x_{n-1},\alpha_n), trans(x_{n-1},a_n), trans(z_{n-1},d_n), rot(z_{n-1},\theta_n)$\\
$\theta_n$ Drehwinkel/Translation um/entlang $z_{n-1}$\\
$\alpha_n$ Verdrehwinkel von $z_{n-1}$ nach $z_n$ um $x_{n-1}$\\
$a_n$ Minimaler Abstand (gemeinsame Normale) zw. $z_{n-1},z_n$ entlang $x_n$\\
$d_n$ Verschiebung der KOSY entlang $z_{n-1}$\\ 

\section{Kinematik}
\begin{tabular}{lll}
Vorwärtskinematik & DIR-KIN & $\vec w = \vec f(\vec q)$\\
Rückwärtskinematik & INV-KIN & $\vec q = \vec g(\vec w)$\\
\end{tabular}



Translatorische Geschwindigkeit: $\vec v = \frac{\diff \vec r}{\diff t}$\\
Rotatorische Geschwindigkeit: 1. $\dot {\vec \Omega} = \frac{\diff }{\diff t} [\Psi, \Theta, \phi]$\\
2. Drehwinkelgeschwindigkeit $\vec \omega$\\

Weltkoordinaten $\vec w$ und Gelenkkoordinaten $\vec q$\\
$\tma{0}{T}{E}(\vec q) = \mat{\tma{0}{R}{E}(\vec q) & \vec r(\vec q) \\ \vec 0^\top & 1}$\\
Weltkoordinaten $\vec w = \mat{ \vec r \\ \vec \Omega} = \vec f(\vec q)$\\
Analytisch: $\vec {\dot w} = \mat{ \dot {\vec r} \\ \dot {\vec \Omega}} = \ma J_{\vec f}(\vec q) \vec{\dot q}$\\
Geometrisch: $\vec {\dot v} = \mat{ \dot {\vec r} \\ \dot {\vec \omega}} = \ma J_{\vec g}(\vec q) \vec{\dot q}$\\
\\
Bewegung von Gelenken:\\
${}_0 \omega_m = {}_0 \omega_{m-1} + {}_0 \ma R{m-1} \cdot \omega_{m-1,m}$\\
${}_0 \dot{\vec r}_m = {}_0 \dot{\vec r}_{m-1} + {}_0 \dot{\vec r}_{m-1,m} + {}_0 \vec \omega_{m-1} \times {}_0 \vec r_{m-1,m}$\\

\begin{tabular}{ll}
Prismatisches Gelenk & Rotatorisches Gelenk \\
${}_0 \vec \omega_{m-1,m} = \vec 0$ & ${}_0 \vec \omega_{m-1,m} =\dot \Theta_m \cdot {}_0 \vec z_m$\\
${}_0 \dot{\vec r}_{m-1,m} = \dot{\ma G}_m \cdot {}_0 \vec z_{m-1}$ & ${}_0 \dot{\vec r}_{m-1,m} = {}_0 \omega_{m-1} \times \vec r_{m-1,m}$\\
\end{tabular}

\begin{tabular}{lll}
& Vorwärtslösung & Rückwärtslösung\\ \mrule
KOSY & $\vec w = \vec f(\vec q)$ & $\vec q = \vec g(\vec w)$\\
Geschw. & $\vec {\dot w} = \ma J_{\vec f}(\vec q) \vec{\dot q}$ & $\vec{\dot q} = \ma J^{-1}_{\vec f}(\vec q) \vec {\dot w}$\\
Momente & $\vec \tau = \ma J^\top(\vec q) \cdot \vec F$ & $\vec F = \left(\ma J^\top(\vec q)\right)^{-1} \cdot \vec \tau$

\end{tabular}


	\subsection{Gelenkmomente}
	$\vec \tau = \ma J^\top(\vec q) \cdot \vec F$


	\subsection{Bahnplanung (Trajektorienplanung)}
	Eine Bahn sollte möglichst weich, als mit geringen Ruck durchfahren werden.\\
	Startpunkt $S$, Zielpunkt $Z$\\
	Durchpunkt $D$ muss mit definierter Geschwindigkeit exakt durchfahren werden.\\
	Viapunkt $V$ muss möglichst nahe passiert werden.\\
	
	Übergangszeit $\tau \ge \max(\tau_{q1}, \tau_{q2}, \cdots, \tau_{qn})$\\
	
	
\section{Manipulatordynamik}
	Lagrangefunktion $L = T(\vec q,\dot{\vec q}) - V(\vec q)$\\
	Euler-Lagrange: $\frac{\diff}{\diff t}  \frac{\partial L}{\partial \dot q_i} - \frac{\partial L}{\partial q_i} = Q_i$\\
	
	\subsection{Modell in Gelenkkoordinaten (Normalform)}
	\boxed{ \ma M(\vec q) \ddot{\vec q} + \vec N(\vec q, \dot{\vec q}) + \vec G(\vec q) + \vec F(\vec q, \dot{\vec q}) = \vec U }\\
	Trägheitsmatrix $\ma M$, Kreiselkräfte $\vec N$, Gravitationskräfte $\vec G$, Dämpfkräfte $\vec F$, Steuerkräfte $\vec U$\\
	
	\subsection{Modell in Weltkoordinaten}
	\boxed{ \ma M_w(\vec q) \ddot{\vec w} + \vec N_w(\vec q, \dot{\vec w}) + \vec G_w(\vec q) + \vec F_w(\vec q, \dot{\vec w}) = \vec U_{\ir Welt} }
	
	
\section{Aktoren}
		\subsection{Harmonic Drive Getriebe}
		Wave Generator(WG), Flexspline(FS), Circular-Spline(CS)\\
	
	
	
\section{Sensoren}
Für innere Zustandsgrößen: Gelenkwinkel\\
Kontaktsensoren im Nahbereich $d \approx \SI{1}{mm}$\\
Annäherungssensoren im Mittelbereich $d < \SI{25}{cm}$\\
Abbildungssensoren im Fernbereich $d > \SI{1}{\meter}$\\
	
	
	
\section{Appoximationstheorie}
Approximationssatz von Stone-Weierstraß: In einem kompakten Intervall kann jede stetige Funktion beliebig genau durch Polynome angenähert werden (Taylorreihe). Jede periodische stetige Funktion kann beliebig genau durch trigonometrische Funktionen angenähert werden. (Fouriereihe)\\
Fehler der Approximation $f = \O(x^n)$

\subsection{Interpolation}
versucht diskrete Punkte durch eine stetige -- und im besten Fall mehrmals differenzierbare -- Funktion zu verbinden. Meist ist das Ziel abprupte Übergänge ``weicher'' und ``glatter'' zu machen.	
Meist wird mit Polynomen interpoliert, da diese leicht zu differenzieren, zu integrieren und auszurechnen sind.
Meist wird zur Bestimmung das Taylorverfahren genutzt. 
Periodische Funktionen lassen sich leichter durch Fourierreihen in eine Summe aus Sinus und Cosinus entwickeln (Trigonometrische Interpolation).
Um die Qualität zu beurteilen, muss der Fehler der Approximation abgeschätzt werden.

\subsection{Ausgleichsrechnung}
versucht die Parameter einer bestimmten Funktion so zu wählen, dass die Funktion möglichst gut diskrete Werte annähert.
Meist wird die Methode der kleinsten Quadrate verwendet. 


\section{Technische Realisierung}
\subsection{Programmierung}
\begin{description}
	\item[Textuell]: Geometriedaten werden am Rechnerterminal programmiert (Offline)
	\item[Grafisch]: Textuell mit Simulator auf CAD-Datenbasis
	\item[Manuell]: Langsames führen der Gelenke in einer Trainingsphase. RAN wird mitgespeichert (Online)
	\item[Sensoriell:] Online Aktualisierung eines Bewegungsablaufs durch Sensordaten.
	\item[Icon-basiert:] ?

\end{description}

\subsection{Antriebskonzepte}
dezentral: Motoren direkt an den Gelenken.\\
zentral: Motoren an der Basis, mechanische Krafübertragung zu den Gelenken.\\

\subsection{Technische Praxis}
Die kinematische Vorwärtslösung sollte mit einer Zykluszeit von $\SI{1}{\milli\second}$ in Gleitkommadarstellung berechnet werden.



\section{Regelungskonzepte}
\begin{itemize}
	\item Lage/Bahnregelung: Kontaktfrei\\
		Vorsteuerung: Sollwerte für Geschwindigkeit und Beschleunigung.
	\item Kraft/Momentregelung: Kontakt mit Umgebung
	\item Hybridregelung: Bahn + Kraftregelung
	\item visuell/Kamerageführte Roboterregelung
\end{itemize}



\section{Maschinelles sehen}
Ortsfilterung mittels Faltung.\\
Frequenzfilterung durch Fouriertransformation\\
Kantendetektion mit Sobel-Operator (extrahiert beliebige Kantenrichtungen)\\
Zusammen mit Laplace Operator Übergänge finden.\\
Bildsegmentierung mit Schwellenwerttabfrage\\
Hough-Transformation: Globale Verbindungsanalyse, findet Geraden.\\

	\subsection{Bildbeschreibung durch Merkmale}
	Drehlageninvariante Merkmale ohne Bezugspunkt.\\
	\begin{itemize}
		\item Objektumfang $U = \int g(x,y) \diff s$
		\item Objektfläche $A = \iint g(x,y) \diff x \diff y$
		\item Lochanzahl, Lochfläche
	\end{itemize}
	
		Mit Bezugspunkt: Schwerpunkt, Inkreis, Umkreis

	\subsection{Bildvergleich durch Template Matching}
	
	Drehlagenbestimmung\\


\section{Praktikum}
Planare Roboter mit Gelenkarmen.\\
Vorwärtslösung: $\vect{r_x \\ r_y \\ \phi } = \vect{\cos(\sum \theta_i)a_n + ... + \cos(\theta_1)a_1 \\ \sin(\sum \theta_i)a_n + ... + \sin(\theta_1)a_1 \\ \sum \theta_i }$\\


% Ende der Spalten
\end{multicols}

% Dokumentende
% ======================================================================
\end{document}


